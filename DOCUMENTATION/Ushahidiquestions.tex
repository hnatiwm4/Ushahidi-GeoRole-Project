\documentclass{article}
\usepackage{graphicx}

\begin{document}

\begin{center}
\textbf{\LARGE Scope and Fact Finding Questions}
\end{center}

\section*{}

\subsection*{What is the reason in limiting the users role to specific geometric regions (the purpose of georoles)?}
\begin{itemize}
\item For specific occupations (like police officer for specific areas they patrol)
\item Is it to reduce the amount of results covered in a given city or region?
\item Will it be used to verify the validity of claims made by users (based on their role)?
\item Could a user have multiple georoles? If so what is your reasoning in giving the same user different roles?
\begin{itemize}
\item If so, will the user be allowed to view more than one georole at a time? 
\end{itemize}
\end{itemize}

\subsection*{What exactly defines a geometric region?}
\begin{itemize}
    \item Should the user/administrator be able to define the exact shape using some sort of tool, or some sort of generic shape tool 
    (circle or square), or based on things like nation, region, city, etc...
\end{itemize}

\subsection*{What are the requirements for a user to gain access to GeoRole features?}
\begin{itemize}
\item What credentials does a user need to possess to access or be restricted to a certain geographic area?
\item Should this feature be setup by the user themselves or by an administrator that handles all user accounts?
\end{itemize}

\subsection*{What do you imagine the interface to setup the georole to look like?}
\begin{itemize}
\item Is it an option where you just click and regions to like a list (adding cities, countries, etc..)?
\item Is it a verbose tool where you can draw/use defined shapes the exact region to encompass in a users georole?
\item Should the user be able to see the defined space of their region (some visual indication that this is their region), 
      or do you show the whole map still and just not show reports from other regions?
\end{itemize}

\subsection*{How is the georole information stored?}
\begin{itemize}
\item Attached to profile?, some external repository? etc...
\item Particular data structure, language or technology?
\end{itemize}
\end{document}
  
  
